An important factor facilitating the application of zebrafish in biomedical research is high throughput screening of vertebrate animal models. Presently, the zebrafish has become a viable model for various research including vertebrate development, gene expression analysis, human diseases modeling, drug screening and toxicology analysis. Compared to other models, the zebrafish embryo is an attractive model due its fast development, small size, transparent embryos that develop outside the mother, and the availability of a large selection of mutant strains. However, the lack of tools for automated analysis of complex images is a huge bottleneck in utilizing the zebrafish to its full potential. 

Zebrafish have a closed circulatory system, and the mechanism of vessel formation are highly similar to those in humans. Being able to model the growth of blood vessel in the vasculature system of zebrafish is interesting for understanding both the circulatory system in humans, and for facilitating large scale screening of the influence of various chemicals on vascular development. Compared to other models, the zebrafish embryo is an attractive alternative for environmental risk assessment of chemicals since it offers the possibility to perform high throughput analysis in vivo. Intersegmental vessels (ISV) and caudal vein plexus (CVP) undergoes active development via angiogenesis. Hence, providing excellent models to study vasculature system.

Most of the current research based on ISV observe the presence or absence of ISVs or perturbation of ISV morphology but do not quantify growth dynamics. Moreover, these analysis are done manually hence it is tedious and expensive. All of these factors facilitate the need for automated image processing methods to quantitatively analyze the image dataset. In this work, we have focused on developing an image processing algorithm to automatically segment and quantify ISVs of zebrafish embryos that have been treated by various toxins. We tested the algorithm on images of zebrafish embryos obtained from screening compounds that may act as an ISV disruptor. The efficiency of segmentation approach is demonstrated by our experiments of the entire zebrafish vasculature recorded from the fluorescence microscope. The experiments also demonstrate that automated segmentation of ISV is comparable to that of manual segmentation. The quantified features are used to train a linear SVM classifier to identify morphological changes in a dataset consisting of ISV zebrafish embryo images. 

%We have also presented work for time-lapse confocal imaging of embryonic vasculature in the zebrafish using image analysis to monitor and quantify the effect of toxins on vascular development.

Recently, studies have indicated that zebrafish CVP also undergoes active development, hence providing an additional measure for studying vascular development. In this work, we have presented an approach to segment and detect abnormalities in the CVP region of zebrafish embryos due to exposure to toxins.  Morphological changes due to toxin exposure is modeled based on the proposed gradient weighted co-occurrence histogram of oriented gradients (gCo-HOG). These features are compared to more commonly used histogram of oriented gradients (HOG) features and Co-HOG features that utilizes spatial distribution of neighboring pixels to capture spatial structure. The features are used to train a linear SVM classifier to identify morphological changes in a dataset of region of CVP zebrafish embryo images.
